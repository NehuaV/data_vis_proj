\documentclass[a4paper,landscape,11pt]{article}
\usepackage[utf8]{inputenc}
\usepackage[T1]{fontenc}
\usepackage{geometry}
\usepackage{graphicx}
\usepackage{float}
\usepackage{xcolor}
\usepackage{setspace}
\usepackage{titlesec}
\usepackage{fancyhdr}
\usepackage{lastpage}
\usepackage{lmodern}

% --- NEW PACKAGES FOR FLEXIBLE LAYOUT ---
\usepackage{paracol}    % Allows parallel columns that break across pages
\usepackage{caption}    % Allows captions outside of float environments
\usepackage{multicol}   % For text-only sections (Tasks 14/15)
\usepackage{hyperref}   % Load hyperref last to avoid conflicts

% Page geometry for landscape A4
\geometry{
    a4paper,
    landscape,
    left=20mm,
    right=20mm,
    top=20mm,
    bottom=20mm,
    marginparwidth=0mm
}

% Hyperref setup
\hypersetup{
    colorlinks=true,
    linkcolor=black,
    urlcolor=blue,
    citecolor=black
}

% Header and footer
\pagestyle{fancy}
\fancyhf{}
\fancyhead[C]{\small Data Visualization Portfolio 2025}
\fancyfoot[C]{\small Page \thepage\ of \pageref{LastPage}}
\renewcommand{\headrulewidth}{0.4pt}
\renewcommand{\footrulewidth}{0.4pt}

% Title formatting
\titleformat{\section}
{\Large\bfseries}
{}
{0em}
{}[\titlerule]

\titleformat{\subsection}
{\large\bfseries}
{}
{0em}
{}

% Spacing
\setlength{\parindent}{0pt}
\setlength{\parskip}{6pt}
\onehalfspacing

% --- LAYOUT CONFIGURATION ---
% Column separation for paracol
\setlength{\columnsep}{2em}

% Custom commands
\newcommand{\tasktitle}[1]{\clearpage\subsection*{Task #1}}

\begin{document}

% --- TITLE PAGE ---
\begin{titlepage}
\centering
\vspace*{2cm}

{\Huge\bfseries Data Visualization Portfolio}\\[1cm]
{\Large Financial Data Analytics}\\[0.5cm]
{\large Dobri Trifonov}\\[0.5cm]
{\large 2025}\\[2cm]

\vfill

{\large \today}

\end{titlepage}

% \newpage
% \tableofcontents
% \newpage

% --- TASK 1 ---
\tasktitle{1: Bad and/or Manipulative Visualization Critique}

\columnratio{0.5}
\begin{paracol}{2}
    % LEFT COLUMN: Image
    \centering
    \includegraphics[width=\linewidth]{../out/1_bad_graph.png}
    \captionof{figure}{Example of a misleading graph from the GPT-5 launch.}
    
    \switchcolumn % Switch to Right Column
    \raggedright
    
    % RIGHT COLUMN: Text
    \textbf{Analysis of Graphical Manipulation}
    
    I analyzed this chart from the OpenAI livestream and found several major issues that make the data hard to trust. The most obvious problem is that the bar heights do not match the numbers provided. For example, the bar for OpenAI o3 claims to be 69.1\%, yet it is visually almost identical in height to the GPT-4o bar, which is only 30.8\%. Since one value is more than double the other, the bars should reflect that difference; instead, this distortion makes the models look more similar than they are to exaggerate the leap made by the new GPT-5 model.

    Furthermore, the legend introduces a confusing inconsistency by only showing the breakdown for ``With thinking'' and ``Without thinking'' modes on the newest model. Because the older models are shown as solid bars, a normal viewer cannot make a fair comparison because it is unclear which mode they are in. This confusion is made worse by the strange ordering of the models. By putting the newest model first rather than following a logical chronological timeline from oldest to newest, the graph feels like it is designed to grab attention rather than provide a clear, honest record of AI improvement.
    
    

\end{paracol}

% --- TASK 2 ---
\tasktitle{2: Improved Version of Bad Visualization}

\columnratio{0.5}
\begin{paracol}{2}
    % LEFT COLUMN: The Improved Image
    \centering
    \includegraphics[width=\linewidth]{../out/2_swe_bench_verified_performance.pdf}
    \captionof{figure}{Improved visualization with corrected scaling and chronological order.}
    
    \switchcolumn 
    \raggedright
    
    % RIGHT COLUMN: Explanation of the Fixes
    \textbf{Implementing the Improvements}

    This version directly addresses the flaws I identified in the original 
    graph to ensure the data is presented honestly and is easy for anyone to follow. 
    I first corrected the scale distortion by ensuring every bar height accurately 
    represents its percentage. By adjusting the y-axis, the massive performance 
    gap between models is now visually clear, showing that the older OpenAI o3 model is 
    actually more than twice as fast compared to the newer GPT-4o model, and that the performance,
    gap is actually caused by the thinking mode. This also highlights that the new GPT-5's 
    biggest performance gains are in the non-thinking mode, which is a surprising result.

    Finally, I rearranged the models into a logical chronological order, 
    starting with the oldest version on the left. This follows the natural way 
    people read progress over time and creates a much more intuitive story about 
    how the technology has evolved.

\end{paracol}


% --- TASK 3 ---
\tasktitle{3: Particularly Good Visualization Critique}

\columnratio{0.5}
\begin{paracol}{2}
    % LEFT COLUMN: The Sankey Diagram
    \centering
    \includegraphics[width=\linewidth]{../out/3_plastic_sankey.png} 
    \captionof{figure}{Sankey diagram by David McCandless showing the flow of plastic waste.}
    
    \switchcolumn 
    \raggedright
    
    % RIGHT COLUMN: Narrative Critique
    \textbf{Why This Visualization Works}
    
    This Sankey diagram about plastic waste is an excellent example of how to tell a complex story through data. It successfully shows the entire lifecycle of plastic products by using paths where the width is proportional to the actual amount of material. This makes it easy for any reader to understand the scale of the problem at a glance, as the massive orange flow for ``Discarded'' creates an immediate visual impact that a simple table of numbers could never achieve.
    
    The chart is particularly effective because it highlights a surprising truth: while 6\% of plastic is labeled as ``Recycled,'' only a tiny 1\% actually makes it ``Back in Use.'' By visually separating these categories, the designer clears up the common misconception that all recycling leads to new products. 

    The visualization is further improved by contextual text snippets, like the fact that half of all plastic was made in just the last twenty years, which adds a sense of urgency to the narrative. Finally, by including both percentages and raw numbers like 8.3 billion tonnes, the chart provides a complete picture that respects both the relative shares and the massive physical scale of global plastic pollution.
    
\end{paracol}


% --- TASK 4 ---
\tasktitle{4: Climate Change Visualization}

\begin{paracol}{1}
    \centering
    \includegraphics[width=0.80\linewidth,height=0.80\textheight,keepaspectratio]{../out/4_climate_change_visualization.pdf}
    \captionof{figure}{Visualization of climate change data}
\end{paracol}


% --- TASK 5 ---
\tasktitle{5: Black-and-White Visualization (No Grey Levels)}

\begin{paracol}{1}
    \centering
    \includegraphics[width=0.80\linewidth,height=0.80\textheight,keepaspectratio]{../out/5_apple_revenue_black_and_white.pdf}
    \captionof{figure}{Black-and-white visualization of Apple's yearly revenue}
\end{paracol}


% --- TASK 6 ---
\tasktitle{6: Visualization Using Color as Important Aesthetics}

\begin{paracol}{1}
    \centering
    \includegraphics[width=0.80\linewidth,height=0.80\textheight,keepaspectratio]{../out/6_ibm_correlation_heatmap.pdf}
    \captionof{figure}{IBM correlation heatmap of Compustat data}
\end{paracol}



% --- TASK 7 ---
\tasktitle{7: Visualization Maximizing Tufte's Data-Ink Ratio}

\begin{paracol}{1}
    \centering
    \includegraphics[width=0.80\linewidth,height=0.80\textheight,keepaspectratio]{../out/7_dataink.pdf}
    \captionof{figure}{Monthly closing prices of IBM stock}
    
\end{paracol}


% --- TASK 8 ---
\tasktitle{8: Non-Standard Visualization Type}

\begin{paracol}{1}
    % LEFT COLUMN
    \centering
    \includegraphics[width=0.80\linewidth,height=0.80\textheight,keepaspectratio]{../out/8_gdp_treemap.pdf}
    \captionof{figure}{Treemap showing global GDP by country (1 trillion USD)}
    
\end{paracol}


% --- TASK 9 ---
\tasktitle{9: Visualization About Yourself}

\begin{paracol}{1}  
    \centering
    \includegraphics[width=0.80\linewidth,height=0.80\textheight,keepaspectratio]{../out/9_entertainment_ratings.pdf}
    \captionof{figure}{My ratings of entertainment media out of 10}
\end{paracol}
    


% --- TASK 10 ---
\tasktitle{10: Data Map}

\begin{paracol}{1}
    \centering
    \includegraphics[width=0.80\linewidth,height=0.80\textheight,keepaspectratio]{../out/10_europe_unemployment_map.pdf}
    \captionof{figure}{Map visualization of European unemployment rates}
\end{paracol}
    

% --- TASK 11 ---
\tasktitle{11: Interactive Visualization}

% This task has 2 images. Paracol allows them to stack naturally in the left column.
\columnratio{0.5}
\begin{paracol}{2}
    % LEFT COLUMN: IMAGES
    \centering
    \includegraphics[width=\linewidth]{../out/11_interactive_unemployment_1.png}
    \vspace{0.5cm} % Space between images
    \includegraphics[width=\linewidth]{../out/11_interactive_unemployment_2.png}
    \captionof{figure}{Screenshots of the interactive European unemployment visualization}
    
    \switchcolumn % SWITCH TO RIGHT COLUMN
    \raggedright
    
    % RIGHT COLUMN: TEXT
    This interactive visualization allows users to explore European unemployment data dynamically.
    
    Users can select different years using a slider, view up-to-date statistics, and see how unemployment rates vary across European countries.
    
    The visualization is deployed as a Streamlit application accessible via web browser.
    
    \textbf{Public URL:} \url{https://data-viz-app.streamlit.app/}

    
\end{paracol}


% --- TASK 12 ---
\tasktitle{12: Visualization Creation Process Documentation (Task 10: Data Map)}

\columnratio{0.5}
\begin{paracol}{2}
    % LEFT COLUMN: Sketches and Versions
    \centering
    \includegraphics[width=\linewidth]{../out/12_idea_1.jpeg}
    \captionof{figure}{Initial sketch focusing on the geographical layout of Europe.}
    
    \switchcolumn 
    \includegraphics[width=\linewidth]{../out/12_idea_2.jpeg}
    \captionof{figure}{Adding color scales and specific numeric labels for better clarity.}
    
\end{paracol}
\newpage

\columnratio{0.5}
\begin{paracol}{2}
    % LEFT COLUMN: Sketches and Versions
    \centering
    \includegraphics[width=\linewidth]{../out/12_idea_3.jpeg}
    \captionof{figure}{Finalizing the design with a legend to handle missing data.}
    
    \switchcolumn 
    \raggedright

    \textbf{From Sketch to Final Map}

    My process for this map started with a simple goal: 
    I wanted to show how unemployment rates vary across Europe at a glance.
     In my first sketch, I focused just on the geography, but I quickly realized that colors alone 
     wouldn't be precise enough for a viewer to understand the exact differences between neighbors. 
     To fix this, I decided to add a color bar for general context and then placed the actual percentage 
     numbers inside each country so the data would be impossible to miss.

    The biggest hurdle appeared when I actually started implementing the code. 
    I found that several countries were missing data in my dataset, which left awkward blank spots on the map. 
    I didn't want viewers to think these countries had 0\% unemployment, 
    so I decided to use a diagonal hatching pattern to represent the missing info. 
    Adding this "No Data" category to the legend was the final step that made the map feel honest and complete.
    
\end{paracol}


% --- TASK 13 ---
\tasktitle{13: Additional Visualizations (Optional)}



\begin{paracol}{1}
    % LEFT COLUMN
    \centering
    \includegraphics[width=0.80\linewidth,height=0.80\textheight,keepaspectratio]{../out/13_word_cloud_tags.pdf}
    \captionof{figure}{Word cloud of tags from MovieLens data}
    
\end{paracol}

\begin{paracol}{1}
    % LEFT COLUMN
    \centering
    \includegraphics[width=0.80\linewidth,height=0.80\textheight,keepaspectratio]{../out/13_movie_count_genre.pdf}
    \captionof{figure}{Bar plot of movie count by genre}
    
\end{paracol}


\begin{paracol}{1}
    % LEFT COLUMN
    \centering
    \includegraphics[width=0.80\linewidth,height=0.80\textheight,keepaspectratio]{../out/13_3d_clusters.pdf}
    \captionof{figure}{3D plot of K-means clusters of MovieLens data}
    
\end{paracol}


% --- TASK 14 ---
\tasktitle{14: Favorite Tools for Data Visualization}

\begin{multicols}{2}

\subsection*{Technical Workflow with Python}

For the technical side of my data projects, 
I rely on a specific set of Python tools to turn raw numbers into a finished story. 
My process always begins with \textbf{Pandas}, which I use for data manipulation and cleaning. 
I find that it makes handling complex math much more readable and faster than writing standard Python code, 
especially when organizing data for the first time. 

Once the data is ready, I switch to \textbf{Seaborn} to create the actual graphics. 
I love it because it offers a high-level interface that produces attractive statistical charts 
without needing a lot of manual styling; it handles professional color palettes and design styles automatically. 

Choosing the right colors is a big part of my design process. I often use \textbf{Set3} for its muted tones, 
which allow me to fill in large blocks of color without the map or chart feeling overwhelming to the viewer. 
When I need something more eye-catching and easier to read at a glance, I switch to the more vibrant \textbf{Set1}. 
In specific cases like bar charts, where I need a wide variety of distinct colors to tell categories apart, 
I prefer \textbf{gist\_ncar} because it provides a very vibrant and clear range. 
Finally, for visualizations that show a scale—like moving from "Positive" to "Negative" values I 
always reach for the \textbf{coolwarm} colormap. 
This sequential palette smoothly transitions from blue (cool) for low or negative values to red (warm) 
for high or positive values, making it intuitive for anyone to spot the data's extremes immediately.



\subsection*{Favorite Data Sources}

Finding high-quality data is the foundation of every visualization. I really like using \textbf{World Bank Data} 
because they provide comprehensive global indicators that are easy to access in several formats, like CSV or Excel. 
For more specialized projects, I enjoy working with \textbf{MovieLens} datasets. 
They offer small versions that are perfect for testing data visualization ideas and machine learning projects, 
but they also provide much larger sets that allow me to verify the scalability and assumptions of my findings. 

\subsection*{AI Tools}
Throughout this entire process, I have integrated AI tools like \textbf{Gemini}. 
It has become a great tool for helping me brainstorm different visualization ideas or quickly generating the initial code I need for a chart. 
It also helps me greatly with debugging and improving the code.

\end{multicols}

\newpage

% --- TASK 15 ---
\tasktitle{15: References}

\begin{multicols}{2}
\section*{Data Sources}
\begin{itemize}
    \item \textbf{European Unemployment Data:} World Bank Data Catalog. \url{https://data.worldbank.org/indicator/SL.UEM.TOTL.ZS} 
    \item \textbf{Map Data:} Natural Earth. \url{https://www.naturalearthdata.com/}
    \item \textbf{MovieLens Data:} MovieLens 100K dataset. \url{https://grouplens.org/datasets/movielens/100k/}
    \item \textbf{Financial Data:} Compustat database. \url{https://wrds-www.wharton.upenn.edu/pages/documents/wrds-data-library/compustat-data/}
    \item \textbf{CRSP Data:} Center for Research in Security Prices. \url{https://wrds-www.wharton.upenn.edu/pages/about/data-vendors/center-for-research-in-security-prices-crsp/}
\end{itemize}

\section*{Visualization Sources}
\begin{itemize}
    \item \textbf{Bad Visualization Example:} OpenAI GPT-5 Reveal Livestream. \url{https://www.youtube.com/live/0Uu_VJeVVfo?si=kcrR9N24rP5IrYFj}
    \item \textbf{Good Visualization Example:} David McCandless, \textit{Information is Beautiful} (2017 data). \url{https://informationisbeautiful.net/visualizations/plastic-crisis-pollution-recycling-microplastics-bioplastics}
\end{itemize}


\subsection*{Additional Software}
\begin{itemize}
    \item LaTeX (for report compilation)
    \item Jupyter Notebook (for data analysis and visualization development)
\end{itemize}

\section*{Generative AI Disclosure}
Generative AI tools were used in the following ways:
\begin{itemize}
    \item Code generation for graphical visualizations for the tasks 2, 4, 5, 6, 7, 8, 9, 10, 11, 12.
    \item LaTeX document structure
\end{itemize}

\end{multicols}

\end{document}